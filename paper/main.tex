%%%%%%%%%%%%%%%%%%%%%%%%%%%%%%%%%%%%%%%%%%%%%%%%%%%%%%%
% Please note that whilst this template provides a 
% preview of the typeset manuscript for submission, it 
% will not necessarily be the final publication layout.
%
% letterpaper/a4paper: US/UK paper size toggle
% num-refs/alpha-refs: numeric/author-year citation and bibliography toggle

%\documentclass[letterpaper]{oup-contemporary}
\documentclass[a4paper,num-refs]{oup-contemporary}

%%% Journal toggle; only specific options recognised.
%%% (Only "gigascience" and "general" are implemented now. Support for other journals is planned.)
\journal{gigascience}

\usepackage{graphicx}
\usepackage{siunitx}

%%% Flushend: You can add this package to automatically balance the final page, but if things go awry (e.g. section contents appearing out-of-order or entire blocks or paragraphs are coloured), remove it!
% \usepackage{flushend}

\title{Computational prediction of human intronic variation}

%%% Use the \authfn to add symbols for additional footnotes, if any. 1 is reserved for correspondence emails; then continuing with 2 etc for contributions.
\author[1,\authfn{1},\authfn{2}]{Pedro Barbosa}
\author[2,\authfn{1},\authfn{2}]{Rosina Savisaar}
\author[2]{Maria Carmo-Fonseca}
\author[2,\authfn{1}]{Alcides Fonseca}

\affil[1]{First Institution}
\affil[2]{Second Institution}

%%% Author Notes
\authnote{\authfn{1}abc@uni.edu; alphbeta@lab.edu}
\authnote{\authfn{2}Contributed equally.}

%%% Paper category
\papercat{Paper}

%%% "Short" author for running page header
\runningauthor{First et al.}

%%% Should only be set by an editor
\jvolume{00}
\jnumber{0}
\jyear{2017}

\begin{document}

\begin{frontmatter}
\maketitle
\begin{abstract}
The Abstract (250 words maximum) should be structured to include the following details: \textbf{Background}, the context and purpose of the study; \textbf{Results}, the main findings; \textbf{Conclusions}, brief summary and potential implications. Please minimize the use of abbreviations and do not cite references in the abstract.
\end{abstract}

\begin{keywords}
Keyword1; keyword 2; keyword 3 (Three to ten keywords representing the main content of the article)
\end{keywords}
\end{frontmatter}

%%% Key points will be printed at top of second page
\begin{keypoints*}
\begin{itemize}
\item This is the first point
\item This is the second point
\item One last point.
\end{itemize}
\end{keypoints*}

\section{Introduction to this Template}

\begin{epigraph}{Epigraph source name}
This is the epigraph text, should you like to add one. Lorem ipsum dolor sit amet, consectetur adipiscing elit, sed do eiusmod tempor incididunt ut labore et dolore magna aliqua. 
\end{epigraph}

This is the \LaTeX{} template for GigaScience journal manuscript submissions. \textbf{Please note that whilst this template provides a preview of the typeset manuscript for submission, it will not necessarily be the final publication layout.}

There are important commands in the preamble that you will need to modify for your own manuscript. If you are using this template on Overleaf, please switch the editor to Source code mode to view them; or if you prefer to stay in the Rich Text view, click on the title in the Rich Text view to display the preamble code.

Use the \verb|\journal{...}| command in the preamble so that the correct journal name, logo and colours are loaded automatically. \textbf{Only certain journals and options are supported at this time;} check with your journal's editorial office if your journal is supported.

Alternatively you can re-define \verb|\jname|, \verb|\jlogo| and the \verb|jcolour| explicitly, though check with your journal's editorial office to confirm that this is appropriate.

Specify your manuscript's category with the \verb|\papercat{...}| command in the preamble.

See the sample code in the preamble for a sample of how author and affiliation information can be specified.

Use later sections starting with `Background' on page~\pageref{sec:background} to write your manuscript. The remainder of this current section will provide some sample \LaTeX{} code for various elements you may want to include in your manuscript.

\subsection{Sectional Headings}
You can use \verb|\section{...}|, \verb|\subsection{...}| commands to add more sections and subsections to your manuscript. Further sectional levels are provided by \verb|\subsubsection|, \verb|\paragraph| and \verb|\subparagraph|.

\subsection{Citations and References}
Use the \verb|num-refs| document class option for numerical citations, and \verb|alph-refs| option for author-year citations.
Use the \verb|\citep| command for parenthetical citations, and \verb|\citet| command for text citations (when using \verb|alpha-refs|).
This is a citation: \citep{Fan:2004} and here are two more: \citep{Cox:1972,Hear:Holm:Step:quan:2006}.

\begin{quote}
This is a quote. Lorem ipsum dolor sit amet, consectetur adipiscing elit, sed do eiusmod tempor incididunt ut labore et dolore magna aliqua. Ut enim ad minim veniam, quis nostrud exercitation ullamco laboris nisi ut aliquip ex ea commodo consequat. 
\end{quote}

\begin{itemize}
\item This is a bullet list.
\item Another point.
\item A third point.
\end{itemize}

This\footnote{This is the footnote text. This is the footnote text. This is the footnote text. This is the footnote text. This is the footnote text.} is a footnote. Lorem ipsum dolor sit amet, consectetur adipiscing elit, sed do eiusmod tempor incididunt ut labore et dolore magna aliqua.

\begin{itemize}
\item This is a numbered list.
\item Another point.
\item A third point.
\end{itemize}

Lorem ipsum dolor sit amet, consectetur adipiscing elit, sed do eiusmod tempor incididunt ut labore et dolore magna aliqua. Lorem ipsum dolor sit amet, consectetur adipiscing elit, sed do eiusmod tempor incididunt ut labore et dolore magna aliqua.


\subsubsection{This is a 3rd level heading}

Use \verb|\subsubsection| to get a 3rd level heading.
Lorem ipsum dolor sit amet, consectetur adipiscing elit, sed do eiusmod tempor incididunt ut labore et dolore magna aliqua. Lorem ipsum dolor sit amet, consectetur adipiscing elit, sed do eiusmod tempor incididunt ut labore et dolore magna aliqua. 


\paragraph{This is a 4th level heading}

Use \verb|\paragraph| to get a 4th level heading.
Lorem ipsum dolor sit amet, consectetur adipiscing elit, sed do eiusmod tempor incididunt ut labore et dolore magna aliqua. Lorem ipsum dolor sit amet, consectetur adipiscing elit. 

\subparagraph{This is a 5th level heading}

Use \verb|\subparagraph| to get a 5th level heading.
Lorem ipsum dolor sit amet, consectetur adipiscing elit, sed do eiusmod tempor incididunt ut labore et dolore magna aliqua. Lorem ipsum dolor sit amet, consectetur adipiscing elit.


\subsection{Figures and Tables}
Figures and tables can be added with the usual \verb|figure| and \verb|table| environments, e.g.~Figure \ref{fig:example} and Table \ref{tab:example}. Use \verb|figure*| and \verb|table*| if you need a two-column wide figure or table, as in Figure \ref{fig:example:wide} and Table \ref{tab:example:wide}. 

If you have a very wide table or figure, you can use \texttt{sidewaystable} or \texttt{sidewaysfigure}, as in Table \ref{tab:example:sideways}: this will be rotated sideways and occupy a \emph{single column} on its own.

If your table or figure is both wide and tall (so it wouldn't fit well in a single column with \texttt{sidewaystable} or \texttt{figure}), 
you can use \verb|table| or \verb|figure| inside a \verb|landscape| environment for a full-page landscaped alternative. A page break will be inserted \emph{immediately before and after} the \verb|landscape| environment (Table \ref{tab:example:landscape}), so you'll need to carefully position it in a suitable location in your manuscript.

\begin{figure}[bt!] %% preferably at bottom or top of column
\centering
\includegraphics[width=\linewidth]{example-image}
\caption{An example figure}\label{fig:example}
\end{figure}

\begin{table}[bt!]
\caption{An example table.}\label{tab:example}
\begin{tabular}{l r l}
\toprule
Item & Quantity & Notes\\
\midrule
Widgets & 42 & Over-supplied\textsuperscript{*} \\
Gadgets & 13 & Under-supplied \\
\bottomrule
\end{tabular}
\begin{tablenotes}
\item This is a table note.
\item \textsuperscript{*}Another note.
\end{tablenotes}
\end{table}


\subsection{Some Mathematics Sample}

Let $X_1, X_2, \ldots, X_n$ be a sequence of independent and identically distributed random variables with $\text{E}[X_i] = \mu$ and $\text{Var}[X_i] = \sigma^2 < \infty$, and let
%
\begin{equation}
S_n = \frac{X_1 + X_2 + \cdots + X_n}{n}
      = \frac{1}{n}\sum_{i}^{n} X_i
\end{equation}
%
denote their mean. Then as $n$ approaches infinity, the random variables $\sqrt{n}(S_n - \mu)$ converge in distribution to a normal $\mathcal{N}(0, \sigma^2)$.


\section{Background}
\label{sec:background}

The background section should be written in a way that is accessible to researchers without specialist knowledge in that area and must clearly state---and, if helpful, illustrate---the background to the research and its aims. The section should end with a brief statement of what is being reported in the article.


\section{Data Description}

A statement providing background and purpose for collection of these data should be presented for readers without specialist knowledge in that area. A brief description of the protocol for data collection, data curation and quality control, as well as potential uses should be included, as well as outlining how the data can be accessed if it is not deposited in our repository.

\begin{figure*}%[b!]  %% Add a [b!] if you prefer the wide image to be at the bottm of the page
\centering
\includegraphics[width=.7\textwidth]{example-image}
\caption{An example wide figure. Lorem ipsum dolor sit amet, consectetur adipiscing elit, sed do eiusmod tempor incididunt ut labore et dolore magna aliqua.
}\label{fig:example:wide}
\end{figure*}


\section{Analyses}

This section should provide details of all of the experiments and analyses that are required to support the conclusions of the paper. The authors should make clear the goal of each analysis and state the basic findings.


\begin{table*}[bt!]
\caption{Automobile land speed records (GR 5-10)}\label{tab:example:wide}
% Use "S" column identifier (from siunitx) to align on decimal point.
% Use "L", "R" or "C" column identifier for auto-wrapping columns with tabularx.
\begin{tabularx}{\linewidth}{S l l l r L}
\toprule
{Speed (mph)} & {Driver} & {Car} & {Engine} & {Date} & {Extra comments}\\
\midrule
407.447     & Craig Breedlove & Spirit of America          & GE J47    & 8/5/63  & (Just to demo a full-width table with auto-wrapping long lines) \\
413.199     & Tom Green       & Wingfoot Express           & WE J46    & 10/2/64  \\
434.22      & Art Arfons      & Green Monster              & GE J79    & 10/5/64  \\
468.719     & Craig Breedlove & Spirit of America          & GE J79    & 10/13/64 \\
526.277     & Craig Breedlove & Spirit of America          & GE J79    & 10/15/65 \\
536.712     & Art Arfons      & Green Monster              & GE J79    & 10/27/65 \\
555.127     & Craig Breedlove & Spirit of America, Sonic 1 & GE J79    & 11/2/65  \\
576.553     & Art Arfons      & Green Monster              & GE J79    & 11/7/65  \\
600.601     & Craig Breedlove & Spirit of America, Sonic 1 & GE J79    & 11/15/65 \\
622.407     & Gary Gabelich   & Blue Flame                 & Rocket    & 10/23/70 \\
633.468     & Richard Noble   & Thrust 2                   & RR RG 146 & 10/4/83  \\
763.035     & Andy Green      & Thrust SSC                 & RR Spey   & 10/15/97\\
\bottomrule
\end{tabularx}

\begin{tablenotes}
\item Source is from this website: \url{https://www.sedl.org/afterschool/toolkits/science/pdf/ast_sci_data_tables_sample.pdf}
\end{tablenotes}
\end{table*}


\section{Discussion}

The discussion should spell out the major conclusions and interpretations of the work, including some explanation on the importance and relevance of the dataset and analysis. It should not be restatement of the analyses done and their basic conclusions. The discussion section can end with a concluding paragraph that clearly states the main conclusions of the research along with directions for future work. Summary illustrations can be included.

\begin{sidewaystable}
\caption{Automobile land speed records (GR 5-10). This is the same table as before, but rotated sideways.}
\label{tab:example:sideways}
% Use "S" column identifier (from siunitx) to align on decimal point.
% Use "L", "R" or "C" column identifier for auto-wrapping columns with tabularx.
\begin{tabularx}{\linewidth}{S l l l r L}
\toprule
{Speed (mph)} & {Driver} & {Car} & {Engine} & {Date} & {Extra comments}\\
\midrule
407.447     & Craig Breedlove & Spirit of America          & GE J47    & 8/5/63  & (Just to demo a full-width table with auto-wrapping long lines) \\
413.199     & Tom Green       & Wingfoot Express           & WE J46    & 10/2/64  \\
434.22      & Art Arfons      & Green Monster              & GE J79    & 10/5/64  \\
468.719     & Craig Breedlove & Spirit of America          & GE J79    & 10/13/64 \\
526.277     & Craig Breedlove & Spirit of America          & GE J79    & 10/15/65 \\
536.712     & Art Arfons      & Green Monster              & GE J79    & 10/27/65 \\
555.127     & Craig Breedlove & Spirit of America, Sonic 1 & GE J79    & 11/2/65  \\
576.553     & Art Arfons      & Green Monster              & GE J79    & 11/7/65  \\
600.601     & Craig Breedlove & Spirit of America, Sonic 1 & GE J79    & 11/15/65 \\
622.407     & Gary Gabelich   & Blue Flame                 & Rocket    & 10/23/70 \\
633.468     & Richard Noble   & Thrust 2                   & RR RG 146 & 10/4/83  \\
763.035     & Andy Green      & Thrust SSC                 & RR Spey   & 10/15/97\\
\bottomrule
\end{tabularx}

\begin{tablenotes}
\item Source is from this website: \url{https://www.sedl.org/afterschool/toolkits/science/pdf/ast_sci_data_tables_sample.pdf}
\end{tablenotes}
\end{sidewaystable}


\section{Potential implications}

Authors should provide some additional comments about potential, more broad-ranging implications of their work that are not directly related to the current focus of their manuscript. This section is meant to promote discussion on possible ways the findings or data presented might be used in or have a relationship with other areas of research that may not be directly apparent in the work. It is not meant to provide `proof of importance' of the work. Only to engender expansion of use to other areas. 

Explicit personal opinions by the authors are permitted, but they should be made clear as such. References or related information to support the propositions should be included. These section should focus on work that can be done within the foreseeable future and specifically using the information within the manuscript, not provide speculation on how it will relate to far-reaching goals of the research area.

\section{Methods}

The methods section should include the design of the study, the type of materials involved, a clear description of all comparisons, and the type of analysis used, to enable replication of the work. Ease of reproducibility is one of the key criteria on which reviewers will be asked to comment, so we strongly advocate the use of the reporting checklists recommended by the \href{https://biosharing.org/}{BioSharing} network and workflow management systems such as \href{https:/galaxyproject.org/}{Galaxy} and \href{http://www.myexperiment.org/home}{MyExperiment} and container systems such as \href{https://www.docker.com/}{Docker}, to save the details of the methods to encourage reproducibility as well as conciseness, is also strongly encouraged. Our \href{http://gigadb.org/}{\textit{Giga}DB repository}{} and \href{http://galaxy.cbiit.cuhk.edu.hk/}{GigaGalaxy} server can also be used to archive data, workflows and snapshots of the code with an accompanying DOI. Our \href{https://github.com/gigascience/papers}{GitHub page} can also be used to host a dynamic forkable version of the code if the authors have not used a code repository themselves.

\emph{GigaScience} encourages and assists with the submission of detailed protocols to the open access repository \href{https://www.protocols.io/}{protocols.io}. Please enter the details into protocols.io, issue a DOI, and cite the protocols.io record from the Methods section.

Authors benefit greatly by posting their methods in protocols.io as these are in a formatted form, allow inclusion of all the details, are fully searchable unlike supplementary files, and can be updated to new versions as basic methodology changes over time. Doing this saves authors extensive time in the future as the methods do not need to be rewritten in future manuscripts as they need only be cited.

\section{Availability of source code and requirements (optional, if code is present)}

Lists the following:
\begin{itemize}
\item Project name: e.g.~My bioinformatics project
\item Project home page: e.g.~\url{http://sourceforge.net/projects/mged}
\item Operating system(s): e.g.~Platform independent
\item Programming language: e.g.~Java
\item Other requirements: e.g.~Java 1.3.1 or higher, Tomcat 4.0 or higher
\item License: e.g.~GNU GPL, FreeBSD etc.
Any restrictions to use by non-academics: e.g. licence needed
\end{itemize}

This needs to be under an \href{http:/opensource.org/licenses}{Open Source Initiative} approved license where practicable compiled running software is made available. If the code is not hosted in a repository the \href{https://github.com/gigascience}{\textit{GigaScience} GitHub repository} is also available for this purpose.

\section{Availability of supporting data and materials}

\textit{GigaScience} requires authors to deposit the data set(s) supporting the results reported in submitted manuscripts in a publicly-accessible data repository such as \href{http://gigadb.org/}{\textit{Giga}DB} (see \textit{Giga}DB database terms of use for complete details). This section should be included when supporting data are available and must include the name of the repository and the permanent identifier or accession number and persistent hyperlinks for the data sets (if appropriate). The following format is recommended:

``The data set(s) supporting the results of this article is(are) available in the [repository name] repository, [cite unique persistent identifier].''

Following the \href{https://www.force11.org/group/joint-declaration-data-citation-principles-final}{Joint Declaration of Data Citation Principles}, where appropriate we ask that the data sets be cited where it is first mentioned in the manuscript, and included in the reference list. If a DOI has been issued to a dataset please always cite it using the DOI rather than the less stable URL the DOI resolves to (e.g.~\url{http://dx.doi.org/10.5524/100044} rather than \url{http://gigadb.org/dataset/100044}). For more see:

Data Citation Synthesis Group: Joint Declaration of Data Citation Principles. Martone M. (ed.) San Diego CA: FORCE11; 2014 [\url{https://www.force11.org/datacitation}]

A list of available scientific research data repositories can be found in \href{http://www.re3data.org/}{res3data} and \href{https://biosharing.org/}{BioSharing}.

\section{Declarations}

\subsection{List of abbreviations}
If abbreviations are used in the text they should be defined in the text at first use, and a list of abbreviations should be provided in alphabetical order.

\subsection{Ethical Approval (optional)}
Manuscripts reporting studies involving human participants, human data or human tissue must:

\begin{itemize}
\item include a statement on ethics approval and consent (even where the need for approval was waived)
\item include the name of the ethics committee that approved the study and the committee's reference number if appropriate
\end{itemize}

Studies involving animals must include a statement on ethics approval and have been treated in a humane manner in line with the \href{http://www.nc3rs.org.uk/arrive-guidelines}{ARRIVE guidelines}.

See our \href{https://academic.oup.com/gigascience/pages/editorial_policies_and_reporting_standards}{editorial policies} for more information.

If your manuscript does not report on or involve the use of any animal or human data or tissue, this section is not applicable to your submission. Please state ``Not applicable'' in this section.

\subsection{Consent for publication}

If your manuscript contains any individual person's data in any form, consent to publish must be obtained from that person, or in the case of children, their parent or legal guardian. All presentations of case reports must have consent to publish. You can use your institutional consent form. You should not send the form to us on submission, but we may request to see a copy at any stage (including after publication). Please also confirm you have followed national guidelines on data collection and release in the place the research was carried out, for example confirming you have Ministry of Science and Technology (MOST) approval in China.

If your manuscript does not contain any individual person's data, please state ``Not applicable'' in this section.

\subsection{Competing Interests}

All financial and non-financial competing interests must be declared in this section. See our \href{https://academic.oup.com/gigascience/pages/editorial_policies_and_reporting_standards}{editorial policies} for a full explanation of competing interests. Where an author gives no competing interests, the listing will read `The author(s) declare that they have no competing interests'. If you are unsure whether you or any of your co-authors have a competing interest please contact the editorial office.


\subsection{Funding}

All sources of funding for the research reported should be declared. The role of the funding body in the design of the study and collection, analysis, and interpretation of data and in writing the manuscript should be declared. Please use \href{http://www.crossref.org/fundingdata/}{FundRef} to report funding sources and include the award/grant number, and the name of the Principal Investigator of the grant. 


\subsection{Author's Contributions}

The individual contributions of authors to the manuscript should be specified in this section. Guidance and criteria for authorship can be found in our \href{https://academic.oup.com/gigascience/pages/editorial_policies_and_reporting_standards}{editorial policies}. We would recommend you follow some kind of standardised taxonomy like the \href{http://docs.casrai.org/CRediT}{CASRAI CRediT} (Contributor Roles Taxonomy).


\section{Acknowledgements}

Please acknowledge anyone who contributed towards the article who does not meet the criteria for authorship including anyone who provided professional writing services or materials.

Authors should obtain permission to acknowledge from all those mentioned in the Acknowledgements section. If you do not have anyone to acknowledge, please write ``Not applicable'' in this section.

See our \href{https://academic.oup.com/gigascience/pages/editorial_policies_and_reporting_standards}{editorial policies} for a full explanation of acknowledgements and authorship criteria.

Group authorship: if you would like the names of the individual members of a collaboration group to be searchable through their individual PubMed records, please ensure that the title of the collaboration group is included on the title page and in the submission system and also include collaborating author names as the last paragraph of the “Acknowledgements” section. Please add authors in the format First Name, Middle initial(s) (optional), Last Name. You can add institution or country information for each author if you wish, but this should be consistent across all authors.

Please note that individual names may not be present in the PubMed record at the time a published article is initially included in PubMed as it takes PubMed additional time to code this information.

\section{Authors' information (optional)}

You may choose to use this section to include any relevant information about the author(s) that may aid the reader's interpretation of the article, and understand the standpoint of the author(s). This may include details about the authors' qualifications, current positions they hold at institutions or societies, or any other relevant background information. Please refer to authors using their initials. Note this section should not be used to describe any competing interests.



%% Specify your .bib file name here, without the extension
\bibliography{paper-refs}

\begin{landscape}
\begin{table}
\caption{Automobile land speed records (GR 5-10). This is again the same table as before, but on a landscaped page. \textbf{Note that a hard page break is inserted immediately before and after \texttt{landscape}}, so you'll need to carefully position such an environment at a suitable location in your manuscript!}
\label{tab:example:landscape}
\begin{tabularx}{\linewidth}{S l l l r L}
\toprule
{Speed (mph)} & {Driver} & {Car} & {Engine} & {Date} & {Extra comments}\\
\midrule
407.447     & Craig Breedlove & Spirit of America          & GE J47    & 8/5/63  & (Just to demo a full-width table with auto-wrapping long lines) \\
413.199     & Tom Green       & Wingfoot Express           & WE J46    & 10/2/64  \\
434.22      & Art Arfons      & Green Monster              & GE J79    & 10/5/64  \\
468.719     & Craig Breedlove & Spirit of America          & GE J79    & 10/13/64 \\
526.277     & Craig Breedlove & Spirit of America          & GE J79    & 10/15/65 \\
536.712     & Art Arfons      & Green Monster              & GE J79    & 10/27/65 \\
555.127     & Craig Breedlove & Spirit of America, Sonic 1 & GE J79    & 11/2/65  \\
576.553     & Art Arfons      & Green Monster              & GE J79    & 11/7/65  \\
600.601     & Craig Breedlove & Spirit of America, Sonic 1 & GE J79    & 11/15/65 \\
622.407     & Gary Gabelich   & Blue Flame                 & Rocket    & 10/23/70 \\
633.468     & Richard Noble   & Thrust 2                   & RR RG 146 & 10/4/83  \\
763.035     & Andy Green      & Thrust SSC                 & RR Spey   & 10/15/97\\
\bottomrule
\end{tabularx}

\begin{tablenotes}
\item Source is from this website: \url{https://www.sedl.org/afterschool/toolkits/science/pdf/ast_sci_data_tables_sample.pdf}
\end{tablenotes}
\end{table}
\end{landscape}



\end{document}
